È in atto, ormai da diversi anni, un piano molto ambizioso che mira alla digitalizzazione
di tutto l'apparato della pubblica amministrazione e alla creazione di portali web che
permettano un accesso semplice e veloce ai suoi servizi offerti al pubblico.

Per \textbf{digitalizzazione} si intende un processo che consiste, essenzialmente,
nella dematerializzazione di documenti concreti in file elettronici che possano essere
consultati da qualsiasi dispositivo e offrono una facilità e velocità d'utilizzo,
una sicurezza e un risparmio che l'equivalente cartaceo non potrà mai ottenere.
\\
Questo processo appena descritto non è però pensabile senza che venga accompagnato
da una completa rivoluzione, sia per quanto riguarda le apparecchiature e le
infrastrutture presenti all'interno dei nostri pubblici uffici, sia per quanto
riguarda la competenza di chi, questi dispositivi, dovrà utilizzarli.

La Commissione Europea, con la recente comunicazione
``2030 Digital Compass: the European Way for the Digital Decade"~\cite{intro-1},
compie un passo strategico  prefiggendosi degli obiettivi ben precisi per realizzare
la vision di una Unione Europea basata sulla crescita di consapevolezza informatica
dei cittadini e sul consolidamento di una leadership tecnologica che permetta
alla società di prosperare al meglio.
\\
Tra i quattro obiettivi definiti appare proprio la digitalizzazione dei servizi pubblici;
questo ci fa capire quanto la questione sia cruciale e come la crescita economica
e tecnologica del nostro paese sia inesorabilmente legata ad essa.
\\
E ciò non è di certo passato inosservato alla nostra classe dirigente,
la quale, nel ``Piano nazionale di ripresa e resilienza", necessario per accedere ai fondi
dell'ormai noto Recovery Plan, ha indicato come obiettivo all'interno della ``Missione n.1"
la digitalizzazione e modernizzazione della pubblica amministrazione, prevedendo di stanziare,
per gli interventi previsti dalla componente, 11,75 miliardi di euro~\cite{intro-2}~\cite{intro-3}.

Tuttavia, nonostante la disponibilità economica per far fronte a questa trasformazione
sia presente, ci sono ancora diversi scogli che il nostro paese deve superare per riuscire
ad arrivare ad un grado di digitalizzazione soddisfacente. Uno fra tutti è la gestione
della complessa, lenta e, talvolta, corrotta macchina della burocrazia italiana.
\\
La soluzione a questo problema è l'attuazione di un processo che tenda a \textbf{sburocratizzare}
gli uffici statali, sostituendo i processi attuali con nuovi processi digitali che siano
all'altezza della missione: si tratterebbe essenzialmente di sviluppare strumenti informatici
che permettano di salvare, validare e condividere informazioni e documenti in maniera sicura.
Una delle caratteristiche fondamentali di questi strumenti dovrà essere necessariamente la
fiducia, in quanto chi riceve quei documenti dovrà essere sicuro che essi non siano stati
manomessi o corrotti.

Lo sviluppo di strumenti del genere generalmente ricade nell'abbracciare
uno dei due paradigmi di progettazione: centralizzato e distribuito.
\\
Il primo si basa essenzialmente sulla concentrazione dei dati e del loro controllo
all'interno di una o più entità centrali. Ciò può essere sicuramente vantaggioso sotto
il punto di vista della rapidità ed efficienza del servizio e del consumo di risorse.
\\
Tuttavia, se stiamo parlando di sicurezza e fiducia è naturale avere qualche perplessità:
non solo stiamo immagazzinando informazioni critiche in un database centralizzato,
potenzialmente vulnerabile ad attacchi e perdita di dati, ma stiamo anche fornendo
tali informazioni a un'entità di cui sarà necessario avere una totale fiducia per quanto
riguarda il corretto mantenimento dei nostri documenti.
\\
Uno strumento fondato su questo paradigma è l'ormai ampiamente utilizzata firma digitale:
la pubblica amministrazione si interfaccia infatti a servizi di terze parti andando a pagare
alte tariffe periodiche e mettendo in mano a queste aziende dati sensibili e la gestione della
verifica di validità e integrità dei loro documenti.
Uno strumento del genere è costoso e ci costringe a fidarci
di organizzazioni esterne, con tutte le conseguenze del caso.

Al contrario, il paradigma distribuito, si basa su una rete di sistemi interconnessi nei
quali le informazioni vengono replicate e sincronizzate in maniera reciproca, il tutto senza
la necessità di un'entità centrale. 
Questa seconda soluzione è generalmente più lenta, onerosa e inefficiente,
ma permette tuttavia di realizzare sistemi con un maggior livello di sicurezza in quanto
l'autenticità dell'informazione registrata non è sostenuta da un'unica entità ma da una
moltitudine di entità indipendenti tra loro.
Così facendo è molto più complicato che si manifestino attacchi o manomissioni
o che, perlomeno, restino inosservati.

Potremmo quindi progettare un sistema software che si basi su un'architettura distribuita
e che permetta non solo di abbattere i costi dei servizi forniti da intermediari che,
strumenti come la firma digitale, portano con loro necessariamente ma anche,
grazie alla natura democratica di questa architettura, di avere massima sicurezza
e fiducia sull'integrità dei propri documenti.

Dovendo andare ad unire una gestione distribuita di documenti con il bisogno, allo stesso 
tempo, di archiviare i metadati che mi permettano di verificarli, si può pensare di andare
ad unire le due tecnologie distribuite più adatte a questi scopi:
rispettivamente Git e blockchain.

Tra le tecnologie distribuite, \textbf{Git} è sicuramente lo standard de facto per la condivisione
e il tracciamento delle modifiche di insiemi di file di qualsiasi natura ed è uno strumento essenziale per
gli sviluppatori di software sia indipendenti che facenti parte di grandi progetti.
\\
Tuttavia, oltre all'essere uno strumento il cui utilizzo è estremamente raro tra
chi non scrive codice informatico a causa della sua natura poco orientata verso
l'utente medio, soffre anche della mancanza di una funzione per la protezione,
registrazione e controllo d'integrità dei propri insiemi di file.

L'aggiunta della \textbf{blockchain}, tramite i suoi meccanismi di registrazione e consenso,
va ad integrare queste funzioni, rendendo questa unione una soluzione ideale per risolvere
i problemi di verifica d'integrità nel mondo digitale di insiemi di documenti.

In questo documento andremo a presentare una possibile implementazione del
progetto software appena descritto, \textbf{PineSU}.

Il resto della tesi sarà strutturata come segue:
\begin{itemize}
    \item Capitolo 2 - \textbf{Concetti preliminari}: Verranno approfondite le tecnologie menzionate nell'introduzione e presentate altre che hanno permesso una corretta ed efficiente implementazione
    \item Capitolo 3 - \textbf{Il Problema e l'Obiettivo}: Verrà formalizzato il problema che andremo ad affrontare e spiegate alcune delle problematiche implementative e come sono state superate
    \item Capitolo 4 - \textbf{Il Software PineSU}: Verranno presentati l'effettiva implementazione, architettura e workflow del software realizzato
    \item Capitolo 5 - \textbf{Dimostrazioni d'uso per il fine preposto}: Esempio pratico del funzionamento del software tramite alcuni esempi che rappresentano situazioni similari a quelle in cui l'utente potrà trovarsi utilizzandolo
    \item Capitolo 6 - \textbf{Conclusioni e Sviluppi futuri}: Si riassumono gli obiettivi raggiunti, alcune criticità e potenziali sviluppi futuri.
\end{itemize}
