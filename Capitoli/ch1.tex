Il passaggio da analogico a digitale, la cosiddetta “digitalizzazione”, è una delle più grandi
rivoluzioni del secolo scorso sotto tutti i punti di vista. Un processo di certo lento e pieno
di difficoltà che è però in grado di ricompensare ampiamente non solo chi riesce a
formalizzarne il procedimento per uno specifico settore, ma anche tutti coloro che
seguiranno ad usufruire delle risorse e i beni che hanno subito questa trasformazione.
E nei settori non manca di certo quello più importante per le società del ventunesimo secolo,
quello lavorativo. E chi sosteneva che la digitalizzazione delle aziende fosse qualcosa su cui
ci si potesse permettere di prendere il proprio tempo è stato costretto a ricredersi con quel
“pedale di accelerazione” che è stata la pandemia di COVID-19.

Costringendoci a stare tutti a casa una delle conseguenze fondamentali è stata,
come ben sappiamo, la spinta involontaria
di miliardi di individui verso metodi tecnologici capaci di assicurare continuità alle proprie
attività lavorative e didattiche nonché utili a fronteggiare le proprie esigenze personali.
Le aziende, sia pubbliche che private, sono state così costrette a reinventare il proprio modo
di lavorare, sia nell’organizzazione del lavoro subordinato che nel modo in cui si rapportano
con terze parti, che siano clienti, utenti o altre aziende. Tuttavia è stato un processo poco
uniforme che ha lasciato molti elementi ancora in forma mista,
uno tra tutti la componente burocratica. 
La burocrazia non è un problema semplice e non si può affrontare con una veloce
scannerizzazione dei documenti in formato PDF o con una produzione dei documenti
in formato elettronico, molti di questi documenti hanno infatti un’importanza notevole
all’interno dell’azienda e, in molti casi, anche valenza legale, bisogna perciò essere
estremamente sicuri che le informazioni scritte su questi documenti
non vengano manomesse o corrotte.

Un grande problema che si affronta di questi tempi e con questi temi estremamente
caldi è come sviluppare strumenti informatici che permettano di salvare e trasferire
informazioni e documenti in maniera sicura in modo tale da accelerare i tempi della
transizione digitale della burocrazia minimizzandone anche i costi, il tutto sempre
tenendo a mente che chi ne usufruisce non deve necessariamente avere conoscenze informatiche
di alto livello.

Le tecnologie principali per la gestione delle informazioni e della loro autenticità
si basano sui due paradigmi di progettazione: centralizzata e distribuita.
La prima si basa essenzialmente sulla concentrazione dei dati, e quindi del potere,
nelle mani di un’unità centrale, ciò può essere sicuramente vantaggioso dal punto di
vista della rapidità dello svolgimento delle operazioni e del consumo di risorse
(sia computazionali, sia di archiviazione, sia energetiche) ma se stiamo parlando di
sicurezza e fiducia è naturale avere qualche perplessità, non solo stiamo immagazzinando
informazioni critiche in un database centralizzato, potenzialmente vulnerabile ad attacchi
e perdita di dati, ma stiamo anche fornendo tali informazioni a un’entità di cui sarà
necessario avere una totale fiducia per quanto riguarda il corretto mantenimento dei
nostri dati. \\
Complementarmente, il secondo paradigma, ovvero quello distribuito, si basa
su una rete di sistemi interconnessi su cui le informazioni vengono replicate e sincronizzate
in maniera reciproca, il tutto senza la necessità di un’entità centrale.
Questa seconda soluzione è sicuramente più lenta, più costosa, più inefficiente,
ma tuttavia più sicura in quanto l’autenticità dell’informazione registrata non è sostenuta
da un’unica unità ma da una moltitudine, così facendo è molto più complicato che si
manifestino attacchi o una manomissioni o, perlomeno, che restino inosservati. 

Tra le tecnologie distribuite, \textbf{Git} è sicuramente lo standard de facto per la condivisione
e il tracciamento delle modifiche di insiemi di file ed è uno strumento essenziale per
gli sviluppatori di software sia indipendenti che facenti parte di grandi progetti.
Tuttavia, oltre all’essere uno strumento il cui utilizzo è estremamente raro tra chi
non scrive codice informatico a causa della sua natura poco orientata verso l’utente medio,
soffre anche della mancanza di una funzione per la protezione, registrazione e controllo
d’integrità dei propri insiemi di file. Queste lacune esistono perché Git non è un software
che si è mai posto obiettivi del genere, tuttavia si può pensare ad un applicativo che,
utilizzando proprio Git, si pone questi obiettivi, dopotutto una funzione che permette
la verifica di insiemi di file tramite un’architettura distribuita e un’interfaccia più
semplice da usare sono due aspetti che potrebbero rendere questa applicazione un potente
strumento per venire in soccorso alla necessità di digitalizzazione della burocrazia.
Rimane però il dilemma di come poter memorizzare le informazioni necessarie alla verifica 
dei documenti utilizzando strutture distribuite, ed è qui che ci viene d’aiuto la giovanissima 
tecnologia della \textbf{blockchain}.
[Scrivere qualcosa sulla blockchain]

Dato il problema esposto e le tecnologie presentate, in questo documento andremo a
presentare il software che implementa una possibile soluzione a questo problema
proprio grazie a Git e alla blockchain, \textbf{PineSU}.

Il resto della tesi sarà strutturata come segue:
\begin{itemize}
    \item Capitolo 2 - \textbf{Concetti preliminari}: Verranno approfondite le tecnologie menzionate nell’introduzione e presentate altre che hanno permesso una corretta ed efficiente implementazione
    \item Capitolo 3 - \textbf{Il Problema e l’Obiettivo}: Verrà formalizzato il problema che andremo ad affrontare e spiegate alcune delle problematiche implementative e come sono state superate
    \item Capitolo 4 - \textbf{Il Software PineSU}: Verranno presentati l’effettiva implementazione, architettura e workflow del software realizzato
    \item Capitolo 5 - \textbf{Dimostrazioni d’uso per il fine preposto}: Esempio pratico del funzionamento del software tramite alcuni esempi che rappresentano situazioni similari a quelle in cui l’utente potrà trovarsi utilizzandolo
    \item Capitolo 6 - \textbf{Conclusioni e Sviluppi futuri}: Si riassumono gli obiettivi raggiunti, alcune criticità e potenziali sviluppi futuri.
\end{itemize}
