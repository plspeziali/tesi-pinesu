
La concretizzazione della soluzione al problema esposto è l’applicativo \textbf{PineSU}. \\
PineSU si presenta come un software leggero scritto in Javascript e che sfrutta il runtime Node.js.
Il software va a considerare gli insiemi di file come delle entità chiamate Storage Unit (SU) in cui
va ad avvolgere logicamente la repository Git. \\
Queste Storage Unit sono le singole unità su cui si andrà poi ad effettuare le singole
operazioni eccetto la registrazione su Blockchain che si svolgerà collettivamente con l’ausilio
di accumulatori crittografici. \\
Vedremo infatti che il ciclo di vita di una SU è scandito dai Blockchain Synchronization Point (BSP).
Quando si decide di andare a registrare il suo stato e la sua presenza sui Blockchain inserendolo,
tramite dei gruppi di suoi simili chiamati Storage Group, nel grande albero la cui radice verrà
salvata effettivamente nella catena immutabile.



\begin{figure}[H]
    \centering
    \resizebox{0.8\textwidth}{!}{
        \begin{tikzpicture}
            \begin{class}[text width=14cm]{MerkleCalendar}{0,0}
                \attribute{}
                \operation{+ addRegistration(name: string, hash: string, date: Date, closed: boolean): void}
                \operation{+ getBSPRoot(hash: string, oHash: string, cHash: string): string}
                \operation{+ getTrees(): [Array, Array]}
            \end{class}
            \begin{class}[text width=8cm]{InternalCalendar}{4,-4.5}
                \attribute{- name : string}
                \attribute{- category : int}
                \attribute{- parent : InternalCalendar}
                \attribute{- category : int}
                \attribute{- hash : string}
                \operation{+ addChild(node: Object) : void}
                \operation{+ calculateHash() : void}
                \operation{+ getChildByName(name: string) : Object}
                \operation{+ findNode(hash: string) : Object}
            \end{class}
            \begin{class}[text width=8cm]{LeafCalendar}{2,-12.5}
                \attribute{- name : string}
                \attribute{- day : int}
                \attribute{- hour : InternalCalendar}
                \attribute{- minute : int}
                \attribute{- hash : string}
                \operation{}
            \end{class}
            \composition{MerkleCalendar}{open, closed}{2}{InternalCalendar};
            %\aggregation{InternalCalendar}{children}{0..*}{LeafCalendar};
            %\aggregation{LeafCalendar}{parent}{0..1}{InternalCalendar};
            \aggregation{[xshift=2cm] InternalCalendar.south}{0..*}{children}{[xshift=2cm] LeafCalendar.north};
            \aggregation{[xshift=-2cm] LeafCalendar.north}{0..1}{parent}{[xshift=-2cm] InternalCalendar.south};
            %\aggregation{[xshift=-2cm] A}{second}{1}{[xshift=-2cm] B.north};
            \selfAssociation{InternalCalendar}{parent}{0,1};
        \end{tikzpicture}
    }
    \caption{M1} \label{fig:M1}
\end{figure}