
Come già affermato in precedenza, il sistema Git,
nonostante la sua completezza e complessità,
non fornisce ai suoi fruitori la possibilità di un controllo d’integrità rigoroso
sulle repository da esso create e gestite, un’operazione essenziale per
alcune organizzazioni dove è cruciale che i dati non vengano manomessi o corrotti. \\
Tali organizzazioni potrebbero aver bisogno di tornare a versioni precedenti
dei loro documenti con la certezza che essi siano stati ripristinati
correttamente oppure di trasferire a terzi insiemi e sottoinsiemi
di file e directory del loro file system facendo si che questi ultimi
possano in ogni momento controllare l’integrità di ogni singolo documento. \\
La possibilità di appoggiarsi ad un sistema centralizzato è a questo punto quella
che offrirebbe una verifica meno sicura: non solo sarebbe necessario avere fiducia
dell'entità che mantiene i propri dati, ma usando dei database centralizzati le informazioni
potrebbero essere eliminate o manomesse con maggiore facilità,
da qui l’idea di appoggiarsi ad una struttura come quella della blockchain (\autoref{sub:bc}).
Tutto ciò deve però essere implementato con un occhio di riguardo alla quantità
di informazioni che verranno salvate su di essa: come sappiamo più byte vogliamo
memorizzare più la nostra operazione sarà costosa dal punto di vista pecuniario.

\section{L’obiettivo del nostro sistema}

Il sistema progettato ha lo scopo di riuscire a fornire a chi ne usufruisce
un livello di astrazione aggiuntivo sopra il software Git tramite un’interfaccia
user-friendly che gli permetta non solo di gestire le sue directory come normali
\emph{repository} (\autoref{sub:vcs}), ma fornisca anche degli utili strumenti di
salvataggio di \emph{hash} (\autoref{sub:hash}) su blockchain, esportazione di sottoinsiemi di repository
e verifica sia di singoli file che di moltitudini.
Tutto ciò implementato con operazioni più o meno severe, a discrezione dell’utente,
permettendo anche di impedire il ricalcolo di determinati insiemi di file con controllo
sullo storico dei \emph{commit} (\autoref{sub:vcs}) o, volendo spendere di più, anche su Blockchain.

\section{La questione della memorizzazione}

Come già menzionato, la quantità di informazioni che andremo a memorizzare
nella blockchain è direttamente proporzionale alla quantità di denaro che
spendiamo per effettuare una registrazione.
Occorre perciò trovare uno stratagemma per poter memorizzare una quantità
estremamente ridotta di dati che permetta però di espletare controlli su un gran numero
di documenti. \\
La soluzione al dilemma arriva sotto forma di accumulatori crittografici (\autoref{sub:mt}).
Utilizzeremo delle strutture ad albero in cui le foglie saranno le nostre unità
informative di interesse e da queste ne ricaveremo una radice unica.
Ovviamente la loro struttura dovrà essere tale da permetterci di andare a reperire
informazioni passate e già calcolate in un tempo che sia relativamente ragionevole.

\newpage