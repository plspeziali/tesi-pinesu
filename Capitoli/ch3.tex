
Come già affermato in precedenza, il sistema Git,
nonostante la sua completezza e complessità,
non fornisce ai suoi fruitori la possibilità di un controllo d'integrità rigoroso
sulle repository da esso create e gestite, un'operazione essenziale per
alcune organizzazioni dove è cruciale che i dati non vengano manomessi o corrotti. \\
Tali organizzazioni potrebbero aver bisogno di tornare a versioni precedenti
dei loro documenti con la certezza che essi siano stati ripristinati
correttamente oppure di trasferire a terzi insiemi e sottoinsiemi
di file e directory del loro file system facendo si che questi ultimi
possano in ogni momento controllare l'integrità di ogni singolo documento. \\
La possibilità di appoggiarsi ad un sistema centralizzato è a questo punto quella
che offrirebbe una verifica meno sicura: non solo sarebbe necessario avere fiducia
dell'entità che mantiene i propri dati, ma usando dei database centralizzati le informazioni
potrebbero essere eliminate o manomesse con maggiore facilità,
da qui l'idea di appoggiarsi ad una struttura come quella della blockchain (\autoref{sub:bc}).
Tutto ciò deve però essere implementato con un occhio di riguardo alla quantità
di informazioni che verranno salvate su di essa: come sappiamo più byte vogliamo
memorizzare più la nostra operazione sarà costosa dal punto di vista pecuniario.
Una corretta soluzione per questo problema con un'interfaccia user-friendly
potrebbe portare enormi benefici soprattutto per quanto riguarda questioni
come il miglioramento del lento e complesso macchinario della burocrazia
all'interno delle aziende pubbliche. 

\section{L'obiettivo del nostro sistema}

Il sistema progettato ha lo scopo di riuscire a fornire a chi ne usufruisce
un livello di astrazione aggiuntivo sopra il software Git tramite un'interfaccia
user-friendly che gli permetta non solo di gestire le sue directory come normali
\emph{repository} (\autoref{sub:vcs}), ma fornisca anche degli utili strumenti di
salvataggio di \emph{hash} (\autoref{sub:hash}) su blockchain, esportazione di sottoinsiemi di repository
e verifica sia di singoli file che di moltitudini.
Tutto ciò implementato con operazioni più o meno severe, a discrezione dell'utente,
permettendo anche di impedire il ricalcolo di determinati insiemi di file con controllo
sullo storico dei \emph{commit} (\autoref{sub:vcs}) o, volendo spendere di più, anche su blockchain.

\section{Perché blockchain?}

Nonostante la spiegazione della natura e del funzionamento della blockchain è comunque
lecito domandarsi se effettivamente questa tecnologia sia la strada giusta e se, a
causa dei suoi costi diretti elevati e della sua lentezza, non sia comunque meglio
affidarsi a delle strutture centralizzate.
Vedremo quindi cinque vantaggi della blockchain e come questi possono andare ad impattare
in maniera molto positiva le problematiche della burocrazia.
Quest'ultima, nell'implementazione classica non digitale, soffre di transazioni talvolta
non controllate e dalla dubbia veridicità, quando esse vengono controllate il prezzo
da pagare è invece elevato: il costo del controllo, di intermediazione da parte di terzi
e di strumenti di certificazione non è sicuramente trascurabile.
Al contrario, la blockchain, presenta le seguenti caratteristiche chiave che cercano di
risolvere questa questione:
\begin{enumerate}
    \item Un registro distribuito che condivide contenuti tra più entità.
    Questa natura condivisa rende le transazioni facilmente tracciabili e
    ampiamente divulgabili anche in ecosistemi larghi e complessi.
    \item La decentralizzazione fisica dell'archivio dei dettagli delle transazioni
    è un elemento di potenziale sicurezza extra intrinseca alla struttura della rete.
    Ciò infatti elimina il rischio del \emph{Single Point of Failure}, dove un singolo
    nodo si trova in uno stato critico e rende la rete vulnerabili agli attacchi informatici.
    \item I nuovi record vengono registrati esclusivamente concatenandoli ai vecchi.
    La loro immutabilità garantisce l'integrità dei dati nel registro.
    \item Le transazioni vengono validate attraverso un meccanismo di consenso peer-to-peer,
    le entità di validazione centrale non sono più necessarie.
    Una conseguenza di questo è lo spostamento del potere dagli intermediari verso
    l'intero ecosistema, la decentralizzazione del potere e del controllo introduce
    il concetto di proprietà dei singoli nodi e permette di mantenere il registro
    bilanciato e verificato tramite le sue metodologie intrinsiche.
    \item Le quattro caratteristiche precedenti permettono la disintermediazione,
    ovvero l'eliminazione di middle-men e intermediatori, eliminando di conseguenza
    anche i loro costi.
\end{enumerate}

\section{La questione della memorizzazione}

Come già menzionato, la quantità di informazioni che andremo a memorizzare
nella blockchain è direttamente proporzionale alla quantità di denaro che
spendiamo per effettuare una registrazione.
Occorre perciò trovare uno stratagemma per poter memorizzare una quantità
estremamente ridotta di dati che permetta però di espletare controlli su un gran numero
di documenti. \\
La soluzione al dilemma arriva sotto forma di accumulatori crittografici (\autoref{sub:mt}).
Utilizzeremo delle strutture ad albero in cui le foglie saranno le nostre unità
informative di interesse e da queste ne ricaveremo una radice unica.
Ovviamente la loro struttura dovrà essere tale da permetterci di andare a reperire
informazioni passate e già calcolate in un tempo che sia relativamente ragionevole.

\newpage