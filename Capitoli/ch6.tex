\label{cap:sf}
Abbiamo potuto osservare come, con il software descritto, si è risolto il problema
della mancanza di una funzione di verifica d’integrità dei file su Git,
andando a sfruttare le tecnologie del Web 3.0 come la Blockchain Ethereum,
sempre con un occhio di riguardo al portafoglio dell’utilizzatore,
servendosi di strutture dati peculiari progettate per permettere un salvataggio economico
e una verifica veloce ed efficiente.

Il progetto ha già subito diverse revisioni e riscritture e si trova ora in uno
stato ben definito e molto fedele alla descrizione fornita
(tralasciando alcuni dettagli poco importanti), tuttavia, nonostante sia un programma
decisamente completo sotto il punto di vista delle funzionalità,
almeno per ciò che avevamo progettato di realizzare, sono presenti alcuni aspetti
su cui si potrebbe ancora lavorare per rendere l’esperienza d’uso molto più adatta
ad affrontare le esigenze lavorative di ogni giorno all’interno di grandi e piccole aziende.

In primis potrebbe essere migliorato il calcolo della Merkle Root corrispondente ad
una singola Storage Unit, infatti per ora la lista di file e directory viene semplicemente
ordinata in ordine alfabetico e da quella viene calcolato un Merkle Tree binario.
Invece, ispirandosi al modo con cui Git traccia le modifiche dei propri file
indentificandoli con il loro hash (come spiegato nella \autoref{sub:git}),
si potrebbe identificare ogni file con l’hash corrispondente e tracciare le modifiche
ogni volta che si effettua un ricalcolo della Storage Unit, evitando quindi di ricalcolare
hash di file che sono rimasti identici, usando possibilmente le funzionalità dei Git commit.

Una seconda aggiunta è sicuramente la creazione del PineSU Smart Contract per la gestione
di registrazioni “forti” su blockchain EVENTUALMENTE POTREI FARLO.

Una terza aggiunta, decisamente più ambiziosa, sarebbe la creazione di una piattaforma
per il salvataggio remoto di Storage Unit, un equivalente a ciò che servizi come GitHub,
BitBucket e GitLab sono per Git.
In questo modo si potrebbe venire incontro anche alle aziende in cui non è
richiesta una grande competenza informatica da parte dei dipendenti
(nonostante l’attuale implementazione di PineSU non si lasci intimidire sotto
il punto di vista dell’essere user-friendly).