\label{cap:sf}
Nell'introduzione abbiamo parlato di come si sentisse la necessità
di un sistema distribuito che permettesse la verifica d'integrità
di insiemi di file e di come la progettazione di uno strumento del
genere potesse portare un grande beneficio al processo di
sburocratizzazione degli enti pubblici.

Ebbene, PineSU, si è dimostrato perfettamente in grado di
compiere rapidamente tutte le operazioni richieste e soddisfare gli
obiettivi che ci eravamo preposti.
L'applicativo riesce infatti a garantire un'esperienza user-friendly andando ad 
accompagnare l'utente nella 
creazione, gestione, registrazione permanente e verifica di insiemi di documenti,
il tutto sfruttando in maniera efficace ed efficiente Git e la blockchain di Ethereum,
due tecnologie che, data la comune natura distribuita e le loro funzionalità,
sono perfette per questo ruolo.

PineSU può infatti vantare, grazie all'implementazione di strutture dati salvate
su descrittori di supporto e metadati e grazie al collegamento remoto esclusivo per la blockchain
e per eventuali repository Git remote scelte dall'utente, una struttura totalmente
decentralizzata e resistente ad attacchi informatici e manomissioni di dati.
Il sistema software ha già subito diverse revisioni e riscritture del codice e si trova ora in uno
stato ben definito; rimangono tuttavia ancora alcuni aspetti da attuare e funzionalità da poter aggiungere.

In primis potrebbe essere migliorato il calcolo della Merkle Root corrispondente
ad una singola Storage Unit, infatti per ora la lista di file e directory viene
semplicemente ordinata in ordine alfabetico e da quella viene calcolato
un Merkle Tree binario. Invece, ispirandosi al modo con cui Git traccia le modifiche
dei propri file (come spiegato nella \autoref{sub:git}),
si potrebbe identificare ogni file con l'hash corrispondente e tracciare le modifiche ogni
volta che si effettua un ricalcolo della Storage Unit, evitando quindi di ricalcolare hash
di file che sono rimasti identici, usando possibilmente le funzionalità dei Git commit.

Una seconda aggiunta, già discussa, è l'implementazione del \emph{PineSU SM}
per registrazioni su blockchain dell'hash di SU chiuse.

Una terza aggiunta è l'implementazione di connettori per ulteriori blockchain, in modo tale
da poter registrare le proprie SU su blockchain pubbliche differenti o, potenzialmente,
anche su blockchain private.

\thispagestyle{mystyle}
Una quarta aggiunta, decisamente più ambiziosa, sarebbe la creazione di una piattaforma
per il salvataggio remoto di Storage Unit, un equivalente a ciò che servizi come GitHub,
BitBucket e GitLab sono per Git, andando ad integrare il salvataggio e la verifica sulle
blockchain pubbliche più gettonate.