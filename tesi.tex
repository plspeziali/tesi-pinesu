\documentclass[12pt,a4paper]{report}
%
% This LaTeX template has been created by Luca Grilli
% Based on the following https://en.wikibooks.org/wiki/LaTeX/Title_Creation
% 
\usepackage[italian]{babel}
%\usepackage[T1]{fontenc} % Riga da commentare se si compila con PDFLaTeX
\usepackage{geometry}
\usepackage{graphicx}
\usepackage{hyperref}
\usepackage[utf8]{inputenc}
\usepackage{lipsum} % genera testo fittizio
\usepackage{subcaption}
\usepackage[nottoc,numbib]{tocbibind}
\usepackage{titlesec}
\usepackage{crimson}
\usepackage[simplified]{pgf-umlcd}
\usepackage{float}
\usepackage{wrapfig,lipsum}
\usepackage{fancyhdr}
\usepackage{listings}
\usepackage{xcolor}
\usepackage{url}
\usepackage{setspace}
\usepackage{longtable}

\newcommand\jsonkey{\color{purple}}
\newcommand\jsonvalue{\color{cyan}}
\newcommand\jsonnumber{\color{orange}}

% switch used as state variable
\makeatletter
\newif\ifisvalue@json

\lstdefinelanguage{json}{
    tabsize             = 4,
    showstringspaces    = false,
    keywords            = {false,true},
    alsoletter          = 0123456789.,
    morestring          = [s]{"}{"},
    stringstyle         = \jsonkey\ifisvalue@json\jsonvalue\fi,
    MoreSelectCharTable = \lst@DefSaveDef{`:}\colon@json{\enterMode@json},
    MoreSelectCharTable = \lst@DefSaveDef{`,}\comma@json{\exitMode@json{\comma@json}},
    MoreSelectCharTable = \lst@DefSaveDef{`\{}\bracket@json{\exitMode@json{\bracket@json}},
    basicstyle          = \ttfamily
}

% enter "value" mode after encountering a colon
\newcommand\enterMode@json{%
    \colon@json%
    \ifnum\lst@mode=\lst@Pmode%
        \global\isvalue@jsontrue%
    \fi
}

% leave "value" mode: either we hit a comma, or the value is a nested object
\newcommand\exitMode@json[1]{#1\global\isvalue@jsonfalse}

\lst@AddToHook{Output}{%
    \ifisvalue@json%
        \ifnum\lst@mode=\lst@Pmode%
            \def\lst@thestyle{\jsonnumber}%
        \fi
    \fi
    %override by keyword style if a keyword is detected!
    \lsthk@DetectKeywords% 
}

\makeatother

\input{solidity-highlighting.tex}

\usetikzlibrary{calc} 
\pagestyle{fancy}

\fancyhf{}
\lhead{\leftmark}
\cfoot{\thepage}

\fontfamily{bch}\selectfont

% Taken from Lena Herrmann at 
% http://lenaherrmann.net/2010/05/20/javascript-syntax-highlighting-in-the-latex-listings-package

\usepackage{color} %use color
\definecolor{mygreen}{rgb}{0,0.6,0}
\definecolor{mygray}{rgb}{0.5,0.5,0.5}
\definecolor{mymauve}{rgb}{0.58,0,0.82}

%Customize a bit the look
\lstset{ %
backgroundcolor=\color{white}, % choose the background color; you must add \usepackage{color} or \usepackage{xcolor}
basicstyle=\footnotesize, % the size of the fonts that are used for the code
breakatwhitespace=false, % sets if automatic breaks should only happen at whitespace
breaklines=true, % sets automatic line breaking
captionpos=b, % sets the caption-position to bottom
commentstyle=\color{mygreen}, % comment style
deletekeywords={...}, % if you want to delete keywords from the given language
escapeinside={<@}{@>}, % if you want to add LaTeX within your code
extendedchars=true, % lets you use non-ASCII characters; for 8-bits encodings only, does not work with UTF-8
frame=single, % adds a frame around the code
keepspaces=true, % keeps spaces in text, useful for keeping indentation of code (possibly needs columns=flexible)
keywordstyle=\color{blue}, % keyword style
% language=Octave, % the language of the code
morekeywords={*,...}, % if you want to add more keywords to the set
numbers=left, % where to put the line-numbers; possible values are (none, left, right)
numbersep=5pt, % how far the line-numbers are from the code
numberstyle=\tiny\color{mygray}, % the style that is used for the line-numbers
rulecolor=\color{black}, % if not set, the frame-color may be changed on line-breaks within not-black text (e.g. comments (green here))
showspaces=false, % show spaces everywhere adding particular underscores; it overrides 'showstringspaces'
showstringspaces=false, % underline spaces within strings only
showtabs=false, % show tabs within strings adding particular underscores
stepnumber=1, % the step between two line-numbers. If it's 1, each line will be numbered
stringstyle=\color{mymauve}, % string literal style
tabsize=2, % sets default tabsize to 2 spaces
title=\lstname, % show the filename of files included with \lstinputlisting; also try caption instead of title
}
%END of listing package%

\definecolor{darkgray}{rgb}{.4,.4,.4}
\definecolor{purple}{rgb}{0.65, 0.12, 0.82}

%define Javascript language
\lstdefinelanguage{JavaScript}{
keywords={typeof, new, true, false, catch, function, return, null, catch, switch, var, if, in, while, do, else, case, break, for, of},
keywordstyle=\color{blue}\bfseries,
ndkeywords={class, export, boolean, throw, implements, import, this},
ndkeywordstyle=\color{darkgray}\bfseries,
identifierstyle=\color{black},
sensitive=false,
comment=[l]{//},
morecomment=[s]{/*}{*/},
commentstyle=\color{purple}\ttfamily,
stringstyle=\color{red}\ttfamily,
morestring=[b]',
morestring=[b]"
}

\lstset{
language=JavaScript,
extendedchars=true,
basicstyle=\footnotesize\ttfamily,
showstringspaces=false,
showspaces=false,
numbers=left,
numberstyle=\footnotesize,
numbersep=9pt,
tabsize=2,
breaklines=true,
showtabs=false,
captionpos=b
}


\titleformat{\chapter}[display]{\Huge\bfseries}{}{0pt}{\thechapter.\ }

\graphicspath{{figures/}}
%
%\addtolength{\topmargin}{-.875in} % reduce the default top margin
%\addtolength{\topmargin}{-2cm} % reduce the default top margin
%

\tolerance=1
\emergencystretch=\maxdimen
\hyphenpenalty=10000
\hbadness=10000

%%%%%%%%%%%%%%%%%%%%%%%%%%%%%%%%%%
%                                %
%     Begin Docuemnt [start]     %
%                                %
%%%%%%%%%%%%%%%%%%%%%%%%%%%%%%%%%%
\begin{document}

\begin{titlepage}

%%%%%%%%%%%%%%%%%%%%%%%%%%%%%%
%     Title Page [start]     %
%%%%%%%%%%%%%%%%%%%%%%%%%%%%%%
% Declare new goemetry for the title page only.
{\Large \noindent Paolo Speziali} \newline

\vspace{1cm}

{\begin{flushleft}
\fontsize{21.8}{26.16} \selectfont \bfseries \noindent 
Condividere informazioni \\
in modo sicuro combinando \\
Git e Blockchain
\end{flushleft}}

\vspace{6mm}

{\large \noindent \emph{Relatore}: \vspace{1.0mm}\\ 
Prof. Luca Grilli \newline}

\vspace{5cm}


\noindent Perugia, Anno Accademico 2020/2021 

\noindent Università degli Studi di Perugia \\
Corso di laurea triennale in Ingegneria Informatica ed Elettronica \\
Dipartimento d'Ingegneria

\vspace{0.7cm}

\noindent \includegraphics[width=0.5\textwidth]{figures/logounipg2021.png}
% Ends the declared geometry for the titlepage
\restoregeometry
\end{titlepage}
\normalfont
%%%%%%%%%%%%%%%%%%%%%%%%%%%%
%     Title Page [end]     %
%%%%%%%%%%%%%%%%%%%%%%%%%%%%
\newpage \thispagestyle{empty} \ \newpage
%%%%%%%%%%%%%%%%%%%%%%%%%%
%     Indice [start]     %
%%%%%%%%%%%%%%%%%%%%%%%%%%
\onehalfspacing
\tableofcontents

%%%%%%%%%%%%%%%%%%%%%%%%
%     Indice [end]     %
%%%%%%%%%%%%%%%%%%%%%%%%

 
\chapter{Introduzione}

Il passaggio da analogico a digitale, la cosiddetta “digitalizzazione”, è una delle più grandi
rivoluzioni del secolo scorso sotto tutti i punti di vista. Un processo di certo lento e pieno
di difficoltà che è però in grado di ricompensare ampiamente non solo chi riesce a
formalizzarne il procedimento per uno specifico settore, ma anche tutti coloro che
seguiranno ad usufruire delle risorse e i beni che hanno subito questa trasformazione.
E nei settori non manca di certo quello più importante per le società del ventunesimo secolo,
quello lavorativo. E chi sosteneva che la digitalizzazione delle aziende fosse qualcosa su cui
ci si potesse permettere di prendere il proprio tempo è stato costretto a ricredersi con quel
“pedale di accelerazione” che è stata la pandemia di COVID-19.

Costringendoci a stare tutti a casa una delle conseguenze fondamentali è stata,
come ben sappiamo, la spinta involontaria
di miliardi di individui verso metodi tecnologici capaci di assicurare continuità alle proprie
attività lavorative e didattiche nonché utili a fronteggiare le proprie esigenze personali.
Le aziende, sia pubbliche che private, sono state così costrette a reinventare il proprio modo
di lavorare, sia nell’organizzazione del lavoro subordinato che nel modo in cui si rapportano
con terze parti, che siano clienti, utenti o altre aziende. Tuttavia è stato un processo poco
uniforme che ha lasciato molti elementi ancora in forma mista,
uno tra tutti la componente burocratica. 
La burocrazia non è un problema semplice e non si può affrontare con una veloce
scannerizzazione dei documenti in formato PDF o con una produzione dei documenti
in formato elettronico, molti di questi documenti hanno infatti un’importanza notevole
all’interno dell’azienda e, in molti casi, anche valenza legale, bisogna perciò essere
estremamente sicuri che le informazioni scritte su questi documenti
non vengano manomesse o corrotte.

Un grande problema che si affronta di questi tempi e con questi temi estremamente
caldi è come sviluppare strumenti informatici che permettano di salvare e trasferire
informazioni e documenti in maniera sicura in modo tale da accelerare i tempi della
transizione digitale della burocrazia minimizzandone anche i costi, il tutto sempre
tenendo a mente che chi ne usufruisce non deve necessariamente avere conoscenze informatiche
di alto livello.

Le tecnologie principali per la gestione delle informazioni e della loro autenticità
si basano sui due paradigmi di progettazione: centralizzata e distribuita.
La prima si basa essenzialmente sulla concentrazione dei dati, e quindi del potere,
nelle mani di un’unità centrale, ciò può essere sicuramente vantaggioso dal punto di
vista della rapidità dello svolgimento delle operazioni e del consumo di risorse
(sia computazionali, sia di archiviazione, sia energetiche) ma se stiamo parlando di
sicurezza e fiducia è naturale avere qualche perplessità, non solo stiamo immagazzinando
informazioni critiche in un database centralizzato, potenzialmente vulnerabile ad attacchi
e perdita di dati, ma stiamo anche fornendo tali informazioni a un’entità di cui sarà
necessario avere una totale fiducia per quanto riguarda il corretto mantenimento dei
nostri dati. \\
Complementarmente, il secondo paradigma, ovvero quello distribuito, si basa
su una rete di sistemi interconnessi su cui le informazioni vengono replicate e sincronizzate
in maniera reciproca, il tutto senza la necessità di un’entità centrale.
Questa seconda soluzione è sicuramente più lenta, più costosa, più inefficiente,
ma tuttavia più sicura in quanto l’autenticità dell’informazione registrata non è sostenuta
da un’unica unità ma da una moltitudine, così facendo è molto più complicato che si
manifestino attacchi o una manomissioni o, perlomeno, che restino inosservati. 

Tra le tecnologie distribuite, \textbf{Git} è sicuramente lo standard de facto per la condivisione
e il tracciamento delle modifiche di insiemi di file ed è uno strumento essenziale per
gli sviluppatori di software sia indipendenti che facenti parte di grandi progetti.
Tuttavia, oltre all’essere uno strumento il cui utilizzo è estremamente raro tra chi
non scrive codice informatico a causa della sua natura poco orientata verso l’utente medio,
soffre anche della mancanza di una funzione per la protezione, registrazione e controllo
d’integrità dei propri insiemi di file. Queste lacune esistono perché Git non è un software
che si è mai posto obiettivi del genere, tuttavia si può pensare ad un applicativo che,
utilizzando proprio Git, si pone questi obiettivi, dopotutto una funzione che permette
la verifica di insiemi di file tramite un’architettura distribuita e un’interfaccia più
semplice da usare sono due aspetti che potrebbero rendere questa applicazione un potente
strumento per venire in soccorso alla necessità di digitalizzazione della burocrazia.
Rimane però il dilemma di come poter memorizzare le informazioni necessarie alla verifica 
dei documenti utilizzando strutture distribuite, ed è qui che ci viene d’aiuto la giovanissima 
tecnologia della \textbf{blockchain}.
[Scrivere qualcosa sulla blockchain]

Dato il problema esposto e le tecnologie presentate, in questo documento andremo a
presentare il software che implementa una possibile soluzione a questo problema
proprio grazie a Git e alla blockchain, \textbf{PineSU}.

Il resto della tesi sarà strutturata come segue:
\begin{itemize}
    \item Capitolo 2 - \textbf{Concetti preliminari}: Verranno approfondite le tecnologie menzionate nell’introduzione e presentate altre che hanno permesso una corretta ed efficiente implementazione
    \item Capitolo 3 - \textbf{Il Problema e l’Obiettivo}: Verrà formalizzato il problema che andremo ad affrontare e spiegate alcune delle problematiche implementative e come sono state superate
    \item Capitolo 4 - \textbf{Il Software PineSU}: Verranno presentati l’effettiva implementazione, architettura e workflow del software realizzato
    \item Capitolo 5 - \textbf{Dimostrazioni d’uso per il fine preposto}: Esempio pratico del funzionamento del software tramite alcuni esempi che rappresentano situazioni similari a quelle in cui l’utente potrà trovarsi utilizzandolo
    \item Capitolo 6 - \textbf{Conclusioni e Sviluppi futuri}: Si riassumono gli obiettivi raggiunti, alcune criticità e potenziali sviluppi futuri.
\end{itemize}


\chapter{Concetti preliminari}

Di seguito si introducono alcuni concetti 
per permettere al lettore di acquisire le nozioni necessarie alla
corretta fruizione del materiale successivo.

\section{Funzioni di hash}
\label{sub:hash}
Una \textbf{funzione di hashing}~\cite{hash} \(h\) è una funzione che permette di associare,
a una qualsiasi sequenza \(m\) di lunghezza arbitraria in input, una sequenza
in output \(h(m)\) di lunghezza costante. 
Questo valore restituito in output è chiamato valore di hash, stringa di hash,
o anche semplicemente \textbf{hash}, mentre il valore preso in input è detto
\textbf{preimmagine}. Possiamo pensare a questa funzione come una ``macchina per
impronte digitali", per ogni sequenza in input essa riesce a calcolarne una stringa binaria
che la identifica univocamente.

Una funzione di hash ha tre caratteristiche fondamentali:
innanzitutto è \emph{deterministica}, ciò significa che per lo stesso input essa deve
generare sempre lo stesso output, deve poi generare esclusivamente
\emph{sequenze in output con una lunghezza fissa}, ciò significa che per qualsiasi input
di qualsiasi lunghezza il risultato dovrà avere sempre una lunghezza di \(b\) bit decisa
a priori, infine deve essere \emph{uniforme}, ovvero i suoi output devono essere
uniformemente distribuiti nel codominio della funzione.
Una stringa di hash, essendo una sequenza binaria, può essere rappresentata in molti modi,
nell'ambito di questo documento presenteremo i vari hash come stringhe esadecimali.

Mentre una funzione di hash generica è tranquillamente utilizzabile per contesti
in cui non è necessaria una particolare sicurezza nel proteggere le caratteristiche
delle preimmagine, quando si ha bisogno che le
informazioni in input rimangano nascoste e si necessita di una maggior sicurezza a scapito
della velocità si ricorre alle \textbf{funzioni crittografiche di hash}.

Una funzione crittografica di hash ha le stesse caratteristiche di una funzione di
hash normale ma aggiunge delle proprietà che deve seguire per poter essere considerata
\emph{crittograficamente sicura}, i valori della sua lunghezza b sono tipicamente
128, 256 e 512, si va quindi ad ottenere degli output potenzialmente molto più lunghi
e che non sembrano adatti alle implementazioni all'interno di semplici strutture dati
per cui le classiche funzioni di hash sono designate.

Le proprietà che permettono di definire una funzione crittografica di hash come sicura sono:
\begin{enumerate}
    \item \emph{Resistenza alla preimmagine}: Dato un hash \(h\) deve essere difficile riuscire a
    trovare un input \(m\) tale che \(h = h(m)\).
    \item \emph{Resistenza alla seconda preimmagine}: Dato un input \(m_1\) deve essere difficile
    riuscire a trovare un diverso input \(m_2\) tale che \(h(m_1) = h(m_2)\).
    \item \emph{Resistenza alla collisione}: Dati due messaggi \(m_1\) ed \(m_2\), deve essere
    difficile che i due messaggi abbiano lo stesso hash, quindi con \(h(m_1) = h(m_2)\).
\end{enumerate}
Da quanto detto si evince che una funzione crittografica di
hash effettua un'operazione unidirezionale: non è possibile (o perlomeno non dovrebbe esserlo),
partendo dal singolo hash, risalire alla preimmagine. \\
Per riuscire a mantenere tali proprietà la funzione, durante la fase di generazione
dell'output, effettua diverse e differenti operazioni sulla preimmagine
che fanno si che anche un solo minuscolo cambiamento all'input generi
un \emph{effetto valanga} sull'output, cambiando radicalmente, se non completamente,
l'hash generato.

Le funzioni crittografiche di hash vengono utilizzate in moltissime implementazioni
nell'ambito della cybersecurity come la verifica di password, la generazione e
validazione di firme digitali e la \textbf{verifica d'integrità di file}.
Quest'ultima assume un'importanza fondamentale anche nel nostro caso: queste funzioni
ci permettono di capire se, dati due file, il loro contenuto è identico senza
la necessità di effettuare alcun controllo byte per byte in quanto produrranno
lo stesso valore di hash, in questo modo possiamo anche capire se un file,
che durante un controllo generava un determinato valore, è stato modificato,
ciò perché il valore generato sarà ovviamente differente.



\section{VCS e Git}
\label{sub:vcs}
Un \textbf{Version Control System}~\cite{vcs1} (o anche VCS), in italiano ``sistema di controllo di versione",
è una tipologia di software per la condivisione,
il controllo e la tracciabilità dei cambiamenti riguardanti determinati file e directory
lungo un lasso di tempo e che permette agli utenti di recuperare rapidamente specifiche
versioni dei loro documenti. Gli insiemi di file e directory gestite da questi sistemi
sono suddivisi in \textbf{repository}, esse rappresentano entità indipendenti tra loro.
Spesso si considera una intera directory di lavoro, con il suo contenuto,
come un'unica repository, potendo però scegliere di escludere alcune risorse.
Un VCS può essere centralizzato o distribuito~\cite{vcs2}.
Nel primo caso è il server centrale che tiene traccia dei cambiamenti e che mantiene e
distribuisce la versione più recente delle risorse richieste, gli utenti possono gestire
le loro repository solo attraverso client lightweight che interagiscono con il server
per riuscire a compiere una qualsiasi operazione.
Nel secondo caso ogni client ha una copia precisa della repository e del suo storico
salvata localmente, i server sono coinvolti solo per effettuare sincronizzazioni
di repository tra i vari client. 

\label{sub:git}
\textbf{Git}~\cite{git-21} è il sistema di controllo di versione distribuito più diffuso al mondo.
Esso modella ogni repository come una \emph{sequenza} o \emph{flusso di snapshot} (istantanee)
di un piccolo file system. Ogni volta che un utente salva lo stato del suo progetto
(tramite l'operazione di \emph{commit}) Git crea uno snapshot di tutti i file e le directory
sotto controllo di versione in quel momento e la archivia nel suo database locale, ogni
file modificato dall'ultimo commit viene incluso nell'ultimo snapshot, mentre i file che
non sono stati modificati non vengono inclusi se non con un collegamento alla loro versione identica
nel commit precedente, in modo da evitare alcuna duplicazione non necessaria.
Ogni risorsa in una repository è identificata internamente dal suo hash (\autoref{sub:hash}) e non dal suo nome,
questo permette a Git di individuare efficientemente i cambiamenti nei file.
Inoltre, quasi ogni operazione di Git va ad aggiungere informazioni al suo database, anche se si tratta
di un'operazione di rimozione, ciò assicura che ogni cambiamento sia reversibile.

Ogni file in una directory assume uno dei questi due stati:
\emph{untracked} (non tracciato) o \emph{tracked} (tracciato).
Un file è \emph{untracked} se non è stato mai aggiunto ad una repository o se è stato
aggiunto ma poi rimosso dalla lista dei file tracciati (comando \textsf{rm}).
Un file \emph{tracked}, ovvero l'esatto opposto di un \emph{untracked}, può assumere a sua volta uno di questi tre
stati: \emph{unmodified} (non modificato o \emph{committed}), \emph{modified} (modificato) e \emph{staged}.
Un file \emph{tracked} è \emph{unmodified} quando coincide con la sua ultima versione nel database.
Se qualsiasi cambiamento avviene, diventa \emph{modified}.
Per diventare \emph{staged} è necessario che l'utente utilizzi su di lui il comando \textsf{add},
in questo modo esso viene viene inserito (o aggiornato se era già presente) nella \emph{staging area}
(o \emph{index}) della repository, essa contiene tutti i file tracciati della repository con una flag
che indica se sono stati modificati o meno dall'ultimo snapshot.
L'operazione di \emph{commit} (comando \textsf{commit}) crea un nuovo snapshot che incorpora
tutti i cambiamenti specificati nella staging area e lo immagazzina nel suo database locale.
A questo punto la staging area verrà ripulita (\emph{cleaned}).
Gli utenti Git possono condividere informazioni e collaborare tra di loro tramite repository remote
su server Git, le quali possono essere sincronizzate con le loro repository locali.

Le operazioni di \emph{pull}, \emph{push}, \emph{clone} e \emph{fetch}
sono tipiche quando si lavora con repository remote.
Il comando \textsf{clone} crea una copia esatta di una repository target,
incluso il suo database di snapshot.
Il comando \textsf{fetch} permette di scaricare le risorse di un progetto remoto che non sono
presenti in quello locale, senza però andare a modificare i file già presenti
applicando eventuali modifiche.
Il comando \textsf{pull} è simile a \textsf{fetch}, eccetto che tenta di eseguire una fusione
automatica del file remoto e del file locale applicando a quest'ultimo le modifiche più recenti.
Infine, il comando \textsf{push} consente di inviare ogni nuovo commit locale al sevrer remoto,
in modo da mantenerli sincronizzati.


\section{Blockchain ed Ethereum}
\label{sub:bc}
Non esiste una definizione formalizzata e universalmente accettata di cosa sia la
\textbf{blockchain}~\cite{block1}~\cite{block2}~\cite{block3}, possiamo però descriverla semplicemente come una lista in continua 
crescita di record, chiamati blocchi, collegati utilizzando metodi crittografici: 
ogni blocco contiene l'hash corrispondente al blocco precedente, un timestamp e i 
dati riguardanti una transazione. 
La \emph{chain}, ovvero la catena, è formata grazie alla presenza del riferimento 
crittografico verso il blocco precedente che si trova all'interno di ogni blocco, 
ciò rende tale catena di blocchi immutabile: il cambiamento dei dati all'interno di 
un blocco andrebbe a far mutare il suo riferimento crittografico, invalidando così la catena.
La blockchain non è controllata all'interno di una singola unità o insieme di 
unità centrali, bensì è distribuita, condivisa e aggiornata tra varie macchine 
all'interno di una rete in modalità \emph{peer-to-peer}.
Nella letteratura scientifica la blockchain è infatti considerata un \textbf{DLT} ~\cite{asv-bdg-19},
Distributed Ledger Technology (Tecnologia di Libro Mastro Distribuito), ovvero
un'infrastruttura unita a dei protocolli che permettono a dei computer locati in
posizioni differenti di proporre e validare transazioni e aggiornare record in maniera
sincronizzata all'interno della rete.

Una transazione, i cui dati sono contenuti all'interno dei blocchi, non è 
altro che un'azione compiuta da parte di un account della rete (i.e. Bob) nei confronti 
di un altro account (i.e. Alice), un esempio classico è l'invio di una quantità di moneta 
virtuale, a quel punto verrà registrata come transazione che il portafoglio o \emph{wallet}
di Bob è in debito di tale quantità nei confronti del portafoglio di Alice.
Possiamo quindi concludere che una transazione altro non è che la registrazione di proprietà 
di un asset, ovvero un elemento, digitale e non, avente valore.

La nascita della blockchain è infatti dovuta alla necessità di avere un ``libro mastro" per 
Bitcoin, un registro che tenesse traccia delle transazioni e che impedisse il fenomeno della 
\emph{doppia spesa}, ovvero ciò che si verifica quando uno stesso titolo valutario viene speso 
due o più volte. Mentre questo fenomeno è controllato in un'economia tradizionale dagli 
istituti finanziari centrali, nell'ambito delle monete digitali distribuite a prendersi 
in carico di effettuare questo controllo fondamentale è la blockchain con il suo 
meccanismo di consenso \emph{Proof-of-Work} 
(meccanismo adottato nella blockchain di Bitcoin, 
ne esistono altri come il \emph{Proof-of-Stake} che sta venendo adottato da Ethereum).

La Proof-of-Work è l'algoritmo di consenso che detta le regole per 
la conferma di transazioni e la produzione di nuovi blocchi della catena: 
per poter aggiungere un blocco è infatti necessario risolvere un complesso 
enigma per cui viene richiesta un'alta potenza computazionale, i \emph{miner} 
o minatori mettono a disposizione le proprie macchine per poter risolvere tale 
enigma in cambio di una ricompensa nella moneta virtuale di tale rete.
Più una catena è lunga, più lavoro computazionale è stato svolto per produrla e 
più gli utenti saranno orientati nel riporre la propria fiducia in essa, soprattutto 
perché un lavoro di manomissione sarebbe a quel punto incredibilmente difficoltoso e costoso.

Vediamo quindi come, nonostante le limitazioni e gli sprechi di risorse che un sistema 
distribuito come la blockchain ha per sua natura, si ha il grande vantaggio 
di un registro condiviso in cui possiamo fidarci della parola di numerosissime 
entità che ne usufruiscono anzichè di un'unica entità centrale.

\label{sub:eth}
Mentre reti come quella Bitcoin a poco si prestano oltre al gestire l'omonima moneta virtuale, 
troviamo altre reti in cui si è pensato di aggiungere molte più funzionalità.
Una di queste è \textbf{Ethereum}~\cite{ethorg-wp-21}~\cite{eth-21}~\cite{eth-22}, una rete che trascende il concetto di semplice rete 
per criptovaluta (in questo caso \textbf{Ether} andando a creare una piattaforma 
decentralizzata per la creazione e la pubblicazione peer-to-peer di Smart Contract 
in linguaggi di programmazione Turing completi (ovvero capaci di risolvere ogni problema 
che gli si possa presentare): Solidity e Vyper. 

Uno \textbf{Smart Contract} è essenzialmente un programma (una collezione di codice che descrive 
funzioni e un insieme di dati) che risiede ad uno specifico indirizzo della blockchain 
(quindi all'interno di uno o più blocchi), gli utilizzatori possono comunicare con loro 
ed utilizzare le loro funzioni richiamando queste ultime tramite delle transazioni. 
Con Ethereum è quindi possibile andare a creare vere e proprie applicazioni decentralizzate
(unendo Smart Contract con un'interfaccia utente), chiamate \textbf{dapp},
con vita propria all'interno della \textbf{Ethereum Virtual Machine}, ovvero
la grande macchina virtuale simulata andando ad unire tutta la potenza computazionale
messa a disposizione dagli utenti di Ethereum e usando la blockchain come archivio
di informazioni permanenti. 

Grazie alla nascita delle dapp si può azzardare ad andare a definire una nuova
tipologia di web, diverso dal Web2 che conosciamo oggi ricco di contenuti generati
dagli utenti ma su piattaforme centralizzate nella mani di poche aziende.
Questo nuovo web, chiamato \textbf{Web3} ha lo scopo di creare una nuova
esperienza di navigazione che permetta a chiunque sia in grado di connettersi
alla rete Ethereum di partecipare al servizio e dove controllo centralizzato,
censura e blocco dei pagamenti sono ormai concetti superati che non appartengono
al grande network decentralizzato, purtroppo tutto ciò a scapito di costi elevati,
spreco di risorse e di una limitata scalabilità, non è detto però che l'evoluzione
di queste tecnologie non permetta di superare questi problemi.

Un grande scoglio per gli sviluppatori di dapp è sicuramente la presenza
delle \textbf{gas fee}, ovvero i dazi sul \textsf{gas}, l'unità di misura del
lavoro computazionale che la EVM compie per svolgere determinate operazioni.
Anche il semplice invio di denaro da un indirizzo a un altro o addirittura anche
solo di un messaggio, poiché implica la creazione di un blocco nella blockchain,
comporta il dovuto pagamento di una tassa. Nel caso di un messaggio registrato in
una transazione, più il messaggio sarà lungo, più la tassa da pagare sarà alta.
L'esistenza delle fee è praticamente obbligata per come la blockchain è strutturata:
il costringere al pagamento ogni qual volta si vuol creare un nuovo blocco rende
la struttura incredibilmente meno vulnerabile allo spamming di nuovi blocchi da
parte di cicli infiniti malevoli o accidentali.
Ciò implica che ogni volta che una dapps vorrà andare a registrare un dato sulla
blockchain qualcuno dovrà pagare affinché ciò accada. 

\section{Node.js}

\textbf{Node.js}~\cite{njs-1}~\cite{njs-2}~\cite{njs-3} è un ambiente di run-time, ovvero che supporta l'esecuzione di
software nonostante non faccia parte del sistema operativo, open source e cross
platform per l'esecuzione di codice in linguaggio Javascript.

\textbf{Javascript} è un linguaggio di programmazione ad alto livello,
spesso con compilazione \emph{Just In Time} e a multi-paradigma: supporta infatti diversi
stili di programmazione tra cui la programmazione a eventi, a oggetti,
funzionale e persino imperativa.
Inizialmente limitato alla sola esecuzione lato client su browser,
la sua crescita in popolarità, complice la sua versatilità, facilità
di apprendimento e la sua sintassi molto simile a C e Java, ha portato
allo sviluppo anche di ambienti esterni su cui poter eseguire codice lato server
e standalone, come il sopracitato Node.js, il quale si basa infatti sull'interprete
Javascript V8, sviluppato da Google per il suo Chrome e dalle prestazioni elevate.

Node.js rappresenta a tutti gli effetti non solo un motore per Javascript server-side,
ma un'implementazione del paradigma \emph{Javascript everywhere}, riuscendo ad unire
tutte le componenti di una qualsiasi applicazione, web e non, sotto un unico linguaggio.

Le applicazioni Node.js vengono eseguite in un singolo processo, senza la creazione di
nuovi thread per ogni richiesta, quando deve essere eseguita un'operazione di I/O,
come una lettura o scrittura nella rete, un accesso a un database o a un file system,
anziché bloccare il thread e sprecare cicli della CPU facendola attendere, Node.js
interrompe il processo e riprende le sue funzioni non appena l'operazione I/O termina,
questo permette a un server con ambiente Node.js di gestire efficientemente moltissime
connessioni in concorrenza senza che il programmatore debba preoccuparsi di gestire
alcuna concorrenza tra thread differenti.

Uno dei punti più affascinanti ed eccitanti di Node.js, e di molti altri linguaggi
con un supporto così attivo da parte della community, è sicuramente il semplice e
veloce accesso alle migliaia di librerie che gli utenti pubblicano ogni giorno e
installabili tramite il package manager \textbf{npm} in maniera pressoché immediata.

\newpage

\section{Accumulatori crittografici e Merkle Tree}
\label{sub:mt}
I \textbf{Merkle Tree}~\cite{mertree} sono una tipologia di \textbf{accumulatori crittografici}, ovvero strumenti che permettono
di comprimere molti elementi informativi in una costante di dimensione fissa, in altre parole
ci permettono di rappresentare più blocchi di dati con un singolo hash.
\begin{wrapfigure}{r}{0.48\textwidth}
    \vspace{-20pt}
    \begin{center}
      \includegraphics[width=0.4\textwidth]{mt1}
    \end{center}
    \vspace{-20pt}
\end{wrapfigure}
I Merkle Tree, nello specifico quelli binari, sono essenzialmente alberi binari
in cui ogni foglia corrisponde all'hash di uno dei nostri elementi, risalendo verso la radice ogni
nodo interno calcolerà il proprio hash concatenando gli hash dei nodi figli, infine si avrà
una radice (\textbf{Merkle Root} o MR) il cui hash è univoco a quella lista di elementi che l'albero
ha come foglie, in quella sequenza.
Inoltre, utilizzando degli hash generati con una funzione crittografica ``forte", si ha
un'assenza di collisioni tra le Merkle Root.
Perciò sappiamo che, per una determinata sequenza di documenti i cui hash sono le foglie dell'albero, anche solo una
piccola modfica ad un file causerebbe un cambiamento significativo, se non totale, della MR.

Possiamo quindi capire che c'è stato un cambiamento, tuttavia per capire anche
quale dei documenti è stato cambiato bisogna ricorrere al concetto di \textbf{Merkle Proof}.
Per effettuare una verifica tramite Merkle Proof sono tre gli elementi necessari:
\begin{enumerate}
    \item L'elemento (foglia) che vogliamo verificare;
    \item La Merkle Root;
    \item La Merkle Proof, ovvero la lista degli hash dei fratelli lungo il cammino dall'elemento alla radice.
\end{enumerate}
Andando a svolgere questa verifica su ogni documento riusceremo ad individuare i file modificati come
quelli per cui non è possibile ricostruire il cammino verso la radice lasciandola inalterata.

\section{JSON}
\label{sub:json}
Il JavaScript Object Notation (\textbf{JSON})~\cite{json} è un semplice formato per lo scambio di dati,
facile da interpretare e capire sia per i vari linguaggi di programmazione che per gli esseri umani.
Esso, con le librerie apposite per ogni linguaggio, permette un semplice e rapido scambio
di dati tra più applicativi e fornisce metodologie per la conversione di oggetti e collezioni
di dati strutturati in stringhe da salvare in file e viceversa, un'alternativa ragionevole ai database
per applicazioni che cercano di sviluppare architetture distribuite.
Osserviamo e analizziamo un esempio di oggetto JSON:
\begin{lstlisting}[language=json,firstnumber=1]
{ "nome": "Mario",
  "cognome": "Rossi",
  "eta": 27,
  "indirizzo": {
    "indirizzoStradale": "Via Fasulla 123",
    "citta": "Perugia"
  },
  "numeriTelefono": [
    {
      "tipo": "casa",
      "numero": "212 555-1234"
    },
    {
      "tipo": "ufficio",
      "numero": "646 555-4567"
    } ],
  "figli": [],
  "coniuge": null }
\end{lstlisting} 
Possiamo osservare come ogni attributo e il suo valore vengano rappresentati come
\[``key" = value \]
dove il valore è posto tra due virgolette in caso sia una stringa (\textsf{riga 1})
o da nulla in caso sia un tipo primitivo (\textsf{riga 3}).
Un valore può inoltre essere anche un altro oggetto (\textsf{righe 4-7}), in questo caso
vediamo come sia incluso tra parentesi graffe e segua poi al suo interno la sintassi che
abbiamo appena osservato, o un array di valori semplici o oggetti (quest'ultimo alle \textsf{righe 8-16})
andando ad inserire i valori tra parentesi quadre.
È possibile anche rappresentare array vuoti (\textsf{riga 17}), con delle parentesi quadre senza
contenuto, e valori nulli (\textsf{riga 18}) con la parola chiave \textsf{null}.
I valori vengono separati tra loro con una semplice virgola.

\chapter{Il Problema e l'Obiettivo}


Come già affermato in precedenza, il sistema Git,
nonostante la sua completezza e complessità,
non fornisce ai suoi fruitori la possibilità di un controllo d'integrità rigoroso
sulle repository da esso create e gestite, un'operazione essenziale per
alcune organizzazioni dove è cruciale che i dati non vengano manomessi o corrotti. \\
Tali organizzazioni potrebbero aver bisogno di tornare a versioni precedenti
dei loro documenti con la certezza che essi siano stati ripristinati
correttamente oppure di trasferire a terzi insiemi e sottoinsiemi
di file e directory del loro file system facendo si che questi ultimi
possano in ogni momento controllare l'integrità di ogni singolo documento. \\
La possibilità di appoggiarsi ad un sistema centralizzato è a questo punto quella
che offrirebbe una verifica meno sicura: non solo sarebbe necessario avere fiducia
dell'entità che mantiene i propri dati, ma usando dei database centralizzati le informazioni
potrebbero essere eliminate o manomesse con maggiore facilità,
da qui l'idea di appoggiarsi ad una struttura come quella della blockchain (\autoref{sub:bc}).
Tutto ciò deve però essere implementato con un occhio di riguardo alla quantità
di informazioni che verranno salvate su di essa: come sappiamo più byte vogliamo
memorizzare più la nostra operazione sarà costosa dal punto di vista pecuniario.
Una corretta soluzione per questo problema con un'interfaccia user-friendly
potrebbe portare enormi benefici soprattutto per quanto riguarda questioni
come il miglioramento del lento e complesso macchinario della burocrazia
all'interno delle aziende pubbliche. 

\section{L'obiettivo del nostro sistema}

Il sistema progettato ha lo scopo di riuscire a fornire a chi ne usufruisce
un livello di astrazione aggiuntivo sopra il software Git tramite un'interfaccia
user-friendly che gli permetta non solo di gestire le sue directory come normali
\emph{repository} (\autoref{sub:vcs}), ma fornisca anche degli utili strumenti di
salvataggio di \emph{hash} (\autoref{sub:hash}) su blockchain, esportazione di sottoinsiemi di repository
e verifica sia di singoli file che di moltitudini.
Tutto ciò implementato con operazioni più o meno severe, a discrezione dell'utente,
permettendo anche di impedire il ricalcolo di determinati insiemi di file con controllo
sullo storico dei \emph{commit} (\autoref{sub:vcs}) o, volendo spendere di più, anche su blockchain.

\section{Perché blockchain?}

Nonostante la spiegazione della natura e del funzionamento della blockchain è comunque
lecito domandarsi se effettivamente questa tecnologia sia la strada giusta e se, a
causa dei suoi costi diretti elevati e della sua lentezza, non sia comunque meglio
affidarsi a delle strutture centralizzate.
Vedremo quindi cinque vantaggi della blockchain e come questi possono andare ad impattare
in maniera molto positiva le problematiche della burocrazia.
Quest'ultima, nell'implementazione classica non digitale, soffre di transazioni talvolta
non controllate e dalla dubbia veridicità, quando esse vengono controllate il prezzo
da pagare è invece elevato: il costo del controllo, di intermediazione da parte di terzi
e di strumenti di certificazione non è sicuramente trascurabile.
Al contrario, la blockchain, presenta le seguenti caratteristiche chiave che cercano di
risolvere questa questione:
\begin{enumerate}
    \item Un registro distribuito che condivide contenuti tra più entità.
    Questa natura condivisa rende le transazioni facilmente tracciabili e
    ampiamente divulgabili anche in ecosistemi larghi e complessi.
    \item La decentralizzazione fisica dell'archivio dei dettagli delle transazioni
    è un elemento di potenziale sicurezza extra intrinseca alla struttura della rete.
    Ciò infatti elimina il rischio del \emph{Single Point of Failure}, dove un singolo
    nodo si trova in uno stato critico e rende la rete vulnerabili agli attacchi informatici.
    \item I nuovi record vengono registrati esclusivamente concatenandoli ai vecchi.
    La loro immutabilità garantisce l'integrità dei dati nel registro.
    \item Le transazioni vengono validate attraverso un meccanismo di consenso peer-to-peer,
    le entità di validazione centrale non sono più necessarie.
    Una conseguenza di questo è lo spostamento del potere dagli intermediari verso
    l'intero ecosistema, la decentralizzazione del potere e del controllo introduce
    il concetto di proprietà dei singoli nodi e permette di mantenere il registro
    bilanciato e verificato tramite le sue metodologie intrinsiche.
    \item Le quattro caratteristiche precedenti permettono la disintermediazione,
    ovvero l'eliminazione di middle-men e intermediatori, eliminando di conseguenza
    anche i loro costi.
\end{enumerate}

\section{La questione della memorizzazione}

Come già menzionato, la quantità di informazioni che andremo a memorizzare
nella blockchain è direttamente proporzionale alla quantità di denaro che
spendiamo per effettuare una registrazione.
Occorre perciò trovare uno stratagemma per poter memorizzare una quantità
estremamente ridotta di dati che permetta però di espletare controlli su un gran numero
di documenti. \\
La soluzione al dilemma arriva sotto forma di accumulatori crittografici (\autoref{sub:mt}).
Utilizzeremo delle strutture ad albero in cui le foglie saranno le nostre unità
informative di interesse e da queste ne ricaveremo una radice unica.
Ovviamente la loro struttura dovrà essere tale da permetterci di andare a reperire
informazioni passate e già calcolate in un tempo che sia relativamente ragionevole.

\newpage

\chapter{Il Software PineSU}


La concretizzazione della soluzione al problema esposto è l’applicativo \textbf{PineSU}. \\
PineSU si presenta come un software leggero scritto in Javascript e che sfrutta il runtime Node.js.
Il software va a considerare gli insiemi di file come delle entità chiamate Storage Unit (SU) in cui
va ad avvolgere logicamente la repository Git. \\
Queste Storage Unit sono le singole unità su cui si andrà poi ad effettuare le singole
operazioni eccetto la registrazione su Blockchain che si svolgerà collettivamente con l’ausilio
di accumulatori crittografici. \\
Vedremo infatti che il ciclo di vita di una SU è scandito dai Blockchain Synchronization Point (BSP).
Quando si decide di andare a registrare il suo stato e la sua presenza sui Blockchain inserendolo,
tramite dei gruppi di suoi simili chiamati Storage Group, nel grande albero la cui radice verrà
salvata effettivamente nella catena immutabile.



\begin{figure}[H]
    \centering
    \resizebox{0.8\textwidth}{!}{
        \begin{tikzpicture}
            \begin{class}[text width=14cm]{MerkleCalendar}{0,0}
                \attribute{}
                \operation{+ addRegistration(name: string, hash: string, date: Date, closed: boolean): void}
                \operation{+ getBSPRoot(hash: string, oHash: string, cHash: string): string}
                \operation{+ getTrees(): [Array, Array]}
            \end{class}
            \begin{class}[text width=8cm]{InternalCalendar}{4,-4.5}
                \attribute{- name : string}
                \attribute{- category : int}
                \attribute{- parent : InternalCalendar}
                \attribute{- category : int}
                \attribute{- hash : string}
                \operation{+ addChild(node: Object) : void}
                \operation{+ calculateHash() : void}
                \operation{+ getChildByName(name: string) : Object}
                \operation{+ findNode(hash: string) : Object}
            \end{class}
            \begin{class}[text width=8cm]{LeafCalendar}{2,-12.5}
                \attribute{- name : string}
                \attribute{- day : int}
                \attribute{- hour : InternalCalendar}
                \attribute{- minute : int}
                \attribute{- hash : string}
                \operation{}
            \end{class}
            \composition{MerkleCalendar}{open, closed}{2}{InternalCalendar};
            %\aggregation{InternalCalendar}{children}{0..*}{LeafCalendar};
            %\aggregation{LeafCalendar}{parent}{0..1}{InternalCalendar};
            \aggregation{[xshift=2cm] InternalCalendar.south}{0..*}{children}{[xshift=2cm] LeafCalendar.north};
            \aggregation{[xshift=-2cm] LeafCalendar.north}{0..1}{parent}{[xshift=-2cm] InternalCalendar.south};
            %\aggregation{[xshift=-2cm] A}{second}{1}{[xshift=-2cm] B.north};
            \selfAssociation{InternalCalendar}{parent}{0,1};
        \end{tikzpicture}
    }
    \caption{M1} \label{fig:M1}
\end{figure}

\chapter{Dimostrazioni d'uso per il fine preposto}

In questo capitolo andremo a testare l'applicativo mostrando cosa accade nel
file system ogni volta che PineSU esegue una funzione.
Prenderemo in esame una situazione in cui creeremo due SU, una verrà lasciata aperta,
l'altra verrà prima registrata aperta, poi chiusa e registrata nuovamente, in questo modo
riusciremo anche ad osservare i cambiamenti sull'intero Merkle Calendar.

\section{Prima inizializzazione}

Alla prima apertura di PineSU sulla macchina il processo ci andrà a chiedere
quattro valori: due indirizzi di Wallet Ethereum, la chiave privata del primo dei wallet e,
opzionalmente, la repository Git remota con cui sincronizzare il MerkleCalendar

\section{Creazione delle Storage Unit}

Partiamo definendo il contenuto delle nostre due directory, la prima,
\textbf{sample}, ha questa struttura:
\begin{itemize}
    \itemsep0em
    \item sample/graphCreator.js
    \item sample/first/astar.js
    \item sample/first/graph.js
    \item sample/second/priorityQueue.js
    \item sample/second/third/main.js
    \item sample/second/third/vertex.js
\end{itemize}
Dove \emph{first} e \emph{second} sono due subdirectory di \emph{sample}
e \emph{third} è una subdirectory di \emph{second}. \\
I file contenuti sono dei file plain text salvati in formato JavaScript. \\ \\
La seconda directory, \textbf{secondSample}, ha questa struttura:
\begin{itemize}
    \itemsep0em
    \item secondSample/esonero1/Immagine.png
    \item secondSample/esonero1/preesonero.pdf
    \item secondSample/esonero1/preesonero.tex
    \item secondSample/esonero2/preesonero2.pdf
    \item secondSample/esonero2/preesonero2.tex
    \item secondSample/esonero3/preesonero3.pdf
    \item secondSample/esonero3/preesonero3.tex
    \item secondSample/esonero3/smith-chart.png
\end{itemize}

Dove \emph{esonero1}, \emph{esonero2} ed \emph{esonero3} sono tre subdirectory di \emph{secondSample}.
In questa directory troviamo anche la presenza di file di diversa natura. \\


Per trasformare le directory in Storage Unit posizioniamoci con il terminale all'interno di ognuna,
avviamo lo script di avvio di PineSU e selezioniamo la prima opzione (\emph{Create / Recalculate SU}),
ovviamente il procedimento andrà ripetuto due volte.
Il processo ci chiederà se inizializzare una repository Git e se escludere alcuni file,
procederà poi al calcolo dei vari hash.

\begin{figure}[H]
    \centering
    \includegraphics[width=0.9\textwidth]{Figures/calculating}
    \caption{\small{
    L'interfaccia di PineSU durante il calcolo.
    } % end small
    } % end caption
    \label{fi:calc}
\end{figure}


Una volta finito di calcolare ci verranno fatte domande sulla natura della Storage Unit come il suo nome,
la repository remota a cui si sincronizza, la sua descrizione, ecc\dots
Finite le domande, l'applicativo ci riporterà al menù principale.

\begin{figure}[H]
    \centering
    \includegraphics[width=0.9\textwidth]{Figures/doneCalculating}
    \caption{\small{
    L'interfaccia di PineSU appena terminata la fase di creazione.
    } % end small
    } % end caption
    \label{fi:dcalc}
\end{figure}

Al termine di entrambe le elaborazioni sulle due directory avremo alcune nuove aggiunte al loro interno:
una cartella .git (eventualmente, anche un file .gitignore) e un file .pinesu.json, quest'ultimo è
il descrittore JSON in cui sono salvate le informazioni che abbiamo inserito, gli hash calcolati e
lo stato di chiusura. Ecco, ad esempio, il file JSON di \emph{sample}:

\begin{lstlisting}[language=json,firstnumber=1]
{"name": "sample",
"remote": "https://github.com/plspeziali/sample",
"description": "A simple Storage Unit",
"visibility": "public",
"date": "2021-08-18",
"owner": "0xCF23544bFC002905532bD86bF647754A84732966",
"hash": "3837ec6fe66032ba593d227ee800a079c61a5853",
"filelist": [
    "first/astar.js:e09deffb3654301a9a8d20acc5a7091cda7039b6",
    "first/graph.js:53982e4feeaf1445434864b409014706e31da1cc",
    "graphCreator.js:c3b4338c33d6ea5a30b31c16defc7661a4ae767b",
    "second/priorityQueue.js:1cb66abaaafd6a7125ab7dac1d7e0fb1860da574",
    "second/third/main.js:1b8dad338691dead8edc66e7c01b8db6d834e3d8",
    "second/third/vertex.js:aa3fa9242ceec7062c7d84764e4068711e53c4e3",
    "first:75d936d52208d14c2cd571e0c595bc29e7d0e3a0",
    "second/third:2ec86afc483f3685893831dfe04b66620be690d2",
    "second:d6e51cdbe8a84cb3ba2c9cfdc2773a96b2401a59"
],
"closed": false }
\end{lstlisting}
\newpage

\section{Staging delle Storage Unit}

A questo punto vogliamo registrare le nostre Storage Unit aperte nella blockchain, 
occorre però prima inserirle negli Storage Group
(in questo caso solo in Open Storage Group).
Per effetturae questa operazione occorro solo selezionare l'opzione omonima
in entrambe le directory, nella nostra cartella d'installazione verrà
aggiornato il file \emph{merkles/storageGroup.json} con le informazioni delle nostre SU.

\begin{lstlisting}[language=json,firstnumber=1]
[ {
    "name": "sample",
    "hash": "3837ec6fe66032ba593d227ee800a079c61a5853",
    "path": "D:/Progetti/Tirocinio/sample",
    "closed": false
  },
  {
    "name": "secondSample",
    "hash": "7a61ed5e43cd436fb1f88895625a8193fdb9b3be",
    "path": "D:/Progetti/Tirocinio/secondSample",
    "closed": false
  } ]  
\end{lstlisting}
Questa è la lista delle foglie con cui calcoleremo l'effettivo Merkle Tree di OSG.

\section{Registrazione su Blockchain}

\chapter{Conclusioni e Sviluppi futuri}

\label{cap:sf}
Nell'introduzione abbiamo parlato di come si sentisse la necessità
di un sistema distribuito che permettesse la verifica d'integrità
di insiemi di file e di come la progettazione di uno strumento del
genere potesse portare un grande beneficio al processo di
sburocratizzazione degli enti pubblici.

Ebbene, PineSU, si è dimostrato perfettamente in grado di
compiere rapidamente tutte le operazioni richieste e soddisfare gli
obiettivi che ci eravamo preposti.
L'applicativo riesce infatti a garantire un'esperienza user-friendly andando ad 
accompagnare l'utente nella 
creazione, gestione, registrazione permanente e verifica di insiemi di documenti,
il tutto sfruttando in maniera efficace ed efficiente Git e la blockchain di Ethereum,
due tecnologie che, data la comune natura distribuita e le loro funzionalità,
sono perfette per questo ruolo.

PineSU può infatti vantare, grazie all'implementazione di strutture dati salvate
su descrittori di supporto e metadati e grazie al collegamento remoto esclusivo per la blockchain
e per eventuali repository Git remote scelte dall'utente, una struttura totalmente
decentralizzata e resistente ad attacchi informatici e manomissioni di dati.
Il sistema software ha già subito diverse revisioni e riscritture del codice e si trova ora in uno
stato ben definito, rimangono tuttavia ancora alcuni aspetti da attuare e funzionalità da poter aggiungere.

In primis potrebbe essere migliorato il calcolo della Merkle Root corrispondente
ad una singola Storage Unit, infatti per ora la lista di file e directory viene
semplicemente ordinata in ordine alfabetico e da quella viene calcolato
un Merkle Tree binario. Invece, ispirandosi al modo con cui Git traccia le modifiche
dei propri file(come spiegato nella \autoref{sub:git}),
si potrebbe identificare ogni file con l'hash corrispondente e tracciare le modifiche ogni
volta che si effettua un ricalcolo della Storage Unit, evitando quindi di ricalcolare hash
di file che sono rimasti identici, usando possibilmente le funzionalità dei Git commit.

Una seconda aggiunta è l'implementazione di connettori per ulteriori blockchain, in modo tale
da poter registrare le proprie SU su blockchain pubbliche differenti o, potenzialmente,
anche su blockchain private.

Una terza aggiunta, decisamente più ambiziosa, sarebbe la creazione di una piattaforma
per il salvataggio remoto di Storage Unit, un equivalente a ciò che servizi come GitHub,
BitBucket e GitLab sono per Git, andando ad integrare il salvataggio e la verifica sulle
blockchain pubbliche più gettonate.


\hyphenpenalty=0
\bibliographystyle{ieeetr}
\bibliography{bibliografia}

\cleardoublepage\phantomsection % to fix wrong hyperref to \part{Epilogue}

\end{document}